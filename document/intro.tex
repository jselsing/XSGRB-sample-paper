
%%%%%%%%%%%%%%%%%%%%%%%%%%%%%%%%%%%%%%%%%%%%%%%%%%%%%%%%%%%%%%%%%%%%%%%%%%%%
\section{Introduction}
%%%%%%%%%%%%%%%%%%%%%%%%%%%%%%%%%%%%%%%%%%%%%%%%%%%%%%%%%%%%%%%%%%%%%%%%%%%%

Only after observing more than 12000 damped Lyman-$\alpha$ absorbers (DLAs)
towards about 10$^5$ QSOs have 5 systems with
$\log({N_\mathrm{HI}/\mathrm{cm^{-2}}}) > 22$ been identified
\citep{Noterdaeme2012}. Long GRB afterglow spectra, by contrast, reveal such
systems in the majority of cases \citep{Jakobsson2006, Fynbo2009}. Whereas DLAs
towards QSOs are mostly limited to $1.8\lesssim z \lesssim 5$ due to the
atmospheric  UV-cutoff and increasing Lyman-blanketing at increasing redshifts
\citep[e.g.,][]{Rafelski2014}, GRBs allow us to see into the hearts of
star-forming galaxies over the full history of cosmic star formation from $z
\approx 0$ to $z > 8$ \citep[e.g.,][]{Tanvir2009, Salvaterra2009,
	Jakobsson2012}. With afterglow spectroscopy (throughout the electromagnetic
spectrum from X-rays to the sub-mm) we can hence characterize the properties of
star-forming galaxies over cosmic history in terms of  redshifts, metallicities,
molecular contents, ISM temperatures, UV-flux densities, etc.. This is, however,
only possible as long as there are satellites in orbit that rapidly and
accurately locate GRBs. The currently operating \textit{Swift} satellite,
launched in 2004 and still fully-functioning, allows for very efficient
follow-up observations of GRBs due to its unprecedented rate, speed and
precision of localisations.

There are several unanswered but fundamental questions that must be addressed in
order to exploit the full potential of GRBs as cosmological probes. More than
50\% of the \textit{Swift} bursts with measured redshift are at $z > 2$, and 
5--10\% are expected to be above $z=5$ \citep{Salvaterra2007,
	Salvaterra2012,Jakobsson2012, Perley2016}. A high redshift completeness is
crucial for our understanding of the link between the number density of GRBs per
unit redshift and the global star-formation history of the Universe, as measured
by other means \citep[UV, FIR, sub-mm, see][]{Robertson2012}. The detection of
GRBs at $z > 6$ shows that GRBs have become competitive as a tool to identifying
galaxies at the highest redshifts and unsurpassed in providing detailed
abundance information via absorption line spectroscopy \citep{Tanvir2012,
	McGuire2016}.

\smallskip

%The sample of short GRBs with redshifts remains too small and too
%incomplete to really
%constrain their redshift distribution (Guetta \& Piran 2006, A\&A, 453, 823;
%Salvaterra et al.\ 2008, MNRAS, 388, 6) making it of extreme importance to
%enlarge as far as possible the number of known redshifts for this class of
%sources.

From March 2005 to March 2016 there have been about 350 {\it Swift} bursts 
complying with
our sample selection criteria (see ``Immediate objective''), and about half of them have
measured redshifts. Among the latter subset, the team proposing these observations has
measured about two thirds of the redshifts, mainly with FORS1/2 and X-shooter (see
Fynbo et al.\ 2009, ApJS, 185, 526;  Jakobsson et al.\ 2012, ApJ, 752, 62;
Kr\"uhler et al.\ 2012, ApJ, 758, 46). Our current aim is to build a sample superior to
our previous low-resolution survey (Fynbo et al.\ 2009, ApJS, 185, 526), both in
terms of quantity and  quality of the spectra. Our program started as 
guaranteed time observations during
periods 84-91, and we have continued in open time since then. 

%In P84 we started the X-shooter legacy survey in GTO time. Our aim is to  build
%a sample of GRBs superior to our previous  low-resolution  survey, Fynbo et al.\
%2009, ApJS, 185, 526) both in terms of quantity and  quality of the spectra. The
%last period of GTO time on the survey was P90 (plus 6 hr of leftover GTO time
%in  P91) and for P92 a proposal like this was  accepted for continuation of the
%program in open time (concerning P93 see box 8). 

\smallskip

X-shooter is in many ways
the ideal GRB follow-up instrument and indeed GRB follow-up was one of the
primary science cases behind the instrument design and implementation. Our
program secures general purpose GRB afterglow spectroscopic follow-up 
that adds strong legacy value to the \textit{Swift} GRB sample.
Due to the wide wavelength range of X-shooter  
with the same observation cover molecular H$_2$ absorption near the
atmospheric cut-off and all the strong emission lines from the host in the
NIR arm (e.g., Friis et al., 2015, MNRAS, 451, 167). In general, the wide wavelength coverage ensures that we always have
features on which to base a redshift measurement as long as the afterglow is
brighter than about 23 mag in either the $R$- or $z$-band. 
Frequently, emission lines are also detected from the underlying host, which
also provide further information such as SFR and metallicity (the top right 
panel in Fig.~1 shows an example). Only for 
7 out of more than 70 secured spectra could we not measure a redshift.   
With the X-shooter survey we
provide {\bf metallicity measurements} for about 30\% (Voigt-profile fits)
of the $z>1.7$ events.
So far we have
measured metallicities for more than 20 GRB afterglows with X-shooter.
With the wide wavelength coverage of X-shooter we can
study important chemical species as Zn\,\textsc{ii}, Cr\,\textsc{ii} and $\alpha$
elements over a much wider redshift range than what is possible with other
instruments.
As an example, we have measured a metallicity of $0.1 Z_\odot$ for GRB\,100219A
at $z=4.669$ (Th{\"o}ne et al.\ 2013, MNRAS, 428, 3590), $0.02Z_\odot$ for 
GRB\,111008A at $z=4.991$ (Sparre et al.\ 2014, ApJ, 785, 150) and $0.05Z_\odot$
for the $z=5.9125$ GRB\,130606A (Hartoog et al.\ 2015, A\&A, 580, 139).  
Reconciling the
abundance patterns of GRB absorbers, other types of absorbers, QSO DLAs
in particular, and old stars in the Local Group is an important long-term
goal (see also Sparre et al.\ 2014, ApJ, 785, 150).
Metallicities are also measured from host emission lines (Kr{\"u}hler et al.\ 2015,
A\&A, 581, A125).
GRB spectroscopy also allows us to determine the dust content of their environments,
both through analysis of the depletion pattern and through measurement of the
associated extinction (Japelj et al.\ 2015, A\&A, 451, 2050). 
This allows us to quantify the dust-to-metals ratio and its
evolution with redshift (e.g., De Cia et al.\ 2013, A\&A, 560, 88; Zafar \&
Watson 2013, A\&A, 560, 26).
%Moreover, GRB
%afterglows allow investigation of random foreground absorbers, complementing
%QSO sightline studies (e.g., Schulze et
%al.\ 2012, A\&A, 546, 20).

%In Fynbo et al. (2009, ApJS, 185,
%526), we compare the host-galaxy HI
%column density and metallicity (most of which has been measured by our group)
%with a sample of damped \lya\ systems (DLAs) along QSO sightlines. We show that
%most GRB-DLAs have a much larger column than QSO-DLAs.  The difference in
%column density and metallicity for GRB and QSO sightlines shows that the total
%cross-section for environments like those of GRBs must be less than a few
%percent of the total cross-section for QSO-DLAs, and suggests that GRB
%sightlines probe regions with high gas density.

%The issue of production and {\bf release of Lyman continuum radiation} is
%currently not fully understood, but the evidence suggests that star-forming
%galaxies are dominating, in particular at the highest redshifts (e.g., Bianchi
%et al.\ 2001, A\&A, 376, 1; Chen et al.\ 2007, ApJ, 667, L25). Also here
%X-shooter spectroscopy will allow major progress due to the excellent blue
%sensitivity.

%GRB afterglows can probe dust in high-redshift star-forming galaxies
%including exploring the {\bf 2175~\AA{} extinction bump}
%(El\'iasd\'ottir et al.\ 2009, ApJ, 697, 1725; Zafar et al.\ 2011, A\&A,
%732, 143).
%Due to the broad wavelength  coverage we can do
%this more accurately and over a broader range of redshifts with X-shooter.

\smallskip

We will also determine the frequency and properties of {\bf molecular
	absorption} towards GRB absorbers. Molecular gas is a key element to catalyze
the process of star formation, but prior to our program H$_2$ had been detected
just in two cases (tentatively in Fynbo et al.\ 2006, A\&A, 451, L47; securely
in Prochaska et al.\ 2009, ApJ, 691, L27). With our X-shooter program we have
found three more systems (Kr\"uhler et al.\ 2013, A\&A, 557, 18; D'Elia et al.\
2014, A\&A, 564, 38; Friis et al.\, 2015, MNRAS, 451, 167). 
We are currently analysing more of our spectra for less obvious molecular
absorption and we expect to find more (a dedicated sample paper is 
addressing this issue).

\vspace{0.2cm}
A natural question to ask is: {\bf how long should this work continue?} Our
view is that we need to keep observing the afterglows as long as we have
\textit{Swift} in operation. Also note that the program is still producing many 
papers and provides data for many theses (Box 9 and 10). \textit{Swift} is currently 
funded until 2018, but is likely to get more extensions given its overwhelming 
success. As
mentioned above GRBs allow us to probe star-forming galaxies that are almost
impossible to study in other ways both in terms of redshifts, galaxy luminosity
function, and regions within galaxies. After 7 periods we have 
secured seven spectra of $z>4$ GRBs, of which three were of sufficient quality to
allow abundance measurements (Th\"one et al.\ 2013, Sparre et al.\ 2014, 
Hartoog et al.\ 2015). GRBs offer the only way to derive chemical abundances 
for the gas phase of central, actively star-forming regions of high-$z$ galaxies. 
The program also maintains a very high discovery
potential where we occasionally find something completely unexpected that 
provides interesting clues to puzzles in other fields, e.g. extinction of 
type Ia supernovae (Fynbo et al.\ 2014, A\&A, 572, 12). Each of these spectra are like
precious jewels – it is a type of observation that can never be repeated and a
class of sightlines that can only be studied while we have operating GRB
satellites. 

It is also worth adding that we have build up a rather unique team spread over
Europe from Granada to Reykjavik, which by now has reached a point where the 
distribution of night shifts, the scientific exploitation of the data is 
efficient and where we are open to all new members who wish to participate.
As mentioned all data are public immediately.

{\bf For all of these reasons, we need to keep
	building up the sample of GRB afterglow spectra now as we may have to wait many
	years before a mission like \textit{Swift} becomes available again.}

A significant proportion of GRBs lack a bright optical afterglow (``{\bf dark bursts}'', e.g.,
Jakobsson et al.\ 2004, ApJ, 617, L21; Melandri et al.\ 2012, MNRAS, 421, 1265). 
Some of these are at the highest
redshifts ($z > 6$) and their observer-frame optical emission is absorbed by
the IGM. The majority, however, suffer from large dust obscuration (e.g., Perley
et al. 2009, AJ, 138, 1690; Greiner et al.\ 2011, A\&A, 526, 30).
Identifying such GRBs is important for
constraining the fraction of obscured star formation. In both cases,
NIR emission is
expected. X-shooter
can adequately study these objects, provided that a NIR counterpart is timely
identified, for which we have the dedicated HAWKI run D. 

The detection of {\bf absorption line variability} can reveal the burst
influence on the surrounding medium and in turn the absorber distance from the
burst and its metallicity (Vreeswijk et al.\ 2007, A\&A 468, 83; D'Elia et al.\
2009, ApJ, 694, 332; Th\"one et al.\ 2011, MNRAS, 414, 479; De Cia et al.\
2012, A\&A, 545, 64; Hartoog et al.\ 2013).
{\bf Short GRBs} originate in a substantially different
environment compared to long GRBs. Short GRBs may be related to the merging of
compact binaries and the coalescence time can be long enough to allow the
progenitor system to move far away from the star formation site (Belczynski et
al.\ 2002, ApJ, 571, 147). Up to now, however, no spectrum with a sufficient 
signal-to-noise ratio
of a short GRB afterglow has been secured.
A knowledge of the redshift distribution
of short bursts is of key importance for the next generation of gravitational
wave experiments, as they are the likely EM counterparts to their
primary targets.


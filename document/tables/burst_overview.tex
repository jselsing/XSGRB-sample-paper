\begin{longtab}
\begin{longtable}{lccccccclc} 
\caption{The full sample of afterglows and hosts observed in the program.
	We here list the burst names and details of the spectroscopic observations. The
	exposure times and slit widths are given in the order UVB/VIS/NIR. The column
	$\Delta t$ shows the time after trigger when the spectroscopic observation was
	started. Mag$_\mathrm{acq}$ gives the approximate magnitude (typically in the
	$R$-band) of the afterglow in the acquisition image. \label{tab:sample_overview}}  \\
\hline\hline
{GRB} &  Obs Date & Exptime & Slit width & Airmass & Seeing & $\Delta t$ & Mag$_\mathrm{acq}$ & Redshift & Notes \\[1.5pt]
\hline
{} & {} &  (ks)   & (arcsec) & {}  &(arcsec) & (hr)   & {} & {} &  \\ [1.5pt]
\hline
\endfirsthead
\caption{The full sample of afterglows or hosts observed in the program (continued).}\\
\hline\hline
{GRB} &  Obs Date & Exptime & Slit width & Airmass & Seeing & $\Delta t$ & Mag$_\mathrm{acq}$ & Redshift & Ref \\[1.5pt]
\hline
{} & {} &  (ks)   & (arcsec) & {}  &(arcsec) & (hr)   & {} & {} &  \\ [1.5pt]
\hline
\endhead
GRB090313\tablefootmark{a} 		                &        2009-03-15         &   6.9/6.9/6.9  	& 1.0/0.9/0.9		& 1.2--1.4  	& 1.5   	& 45      	&  21.6    	& 3.3736 		& (1) \\
GRB090530\tablefootmark{a}  	                &        2009-05-30         &   4.8/4.8/4.8  	& 1.0/1.2/1.2		& 1.6--2.2  	& 1.7   	& 20.6      &  22    	& 1.266 		& (2) \\
GRB090809\tablefootmark{a} 		                &        2009-08-10         &   7.2/7.2/7.2  	& 1.0/0.9/0.9		& 1.2--1.1  	& 1.1   	& 10.2      &  21    	& 2.737  		& (2,3) \\
GRB090926\tablefootmark{a}  	                &        2009-09-27         &   7.2/7.2/7.2  	& 1.0/0.9/0.9		& 1.4--1.5  	& 0.7   	& 22      	&  17.9    	& 2.1062 		& (4) \\
GRB091018     		                            &        2009-10-18         &   2.4/2.4/2.4  	& 1.0/0.9/0.9		& 2.1--1.8  	& 1.0   	& 3.5      	&  19.1    	& 0.9710 		& (5) \\
GRB091127     		                            &        2009-12-02         &   6.0/6.0/6.0  	& 1.0/0.9/0.9		& 1.1--1.2  	& 1.0   	& 101      	&  21.2    	& 0.490  		& (6) \\
GRB100205A     		                            &        2010-02-08         &   10.8/10.8/10.8 	& 1.0/0.9/0.9		& 1.9--1.8  	& 0.9   	& 71      	&   >24    	&  --    		& (2) \\
GRB100219A     		                            &        2010-02-20         &    4.8/4.8/4.8	& 1.0/0.9/0.9		& 1.3--1.1  	& 0.8   	& 12.5      &  23    	& 4.667  		& (7) \\
GRB100316B     		                            &        2010-03-16         &    2.4/2.4/2.4	& 1.0/0.9/0.9		& 2.0--2.4  	& 0.6   	& 0.7      	&  18.2    	& 1.18   		& (2) \\
GRB100316D-1\tablefootmark{b}	                &        2010-03-17         &    3.6/3.6/3.6	& 1.0/0.9/0.9		& 1.2--1.3  	& 0.8   	& 10      	&  21.5     & 0.059  		& (8) \\
GRB100316D-2   		                            &        2010-03-19         &    2.4/2.4/2.4	& 1.0/0.9/0.9		& 1.1--1.2  	& 0.9   	& 58      	&  20.2     & 0.059  		& (8) \\
GRB100316D-3   		                            &        2010-03-20         &    2.6/2.6/3.2	& 1.0/0.9/0.9		& 1.1--1.2  	& 1.1   	& 79      	&  19.9     & 0.059  		& (8) \\
GRB100316D-4   		                            &        2010-03-21         &    2.6/2.6/3.2	& 1.0/0.9/0.9		& 1.1--1.2  	& 1.5   	& 101      	&  19.9     & 0.059  		& (8) \\
GRB100418A-1   		                            &        2010-04-19         &    4.8/4.8/4.8	& 1.0/0.9/0.9		& 1.6--1.3  	& 0.7   	& 8.4      	&  18.1    	& 0.624 		& (9) \\
GRB100418A-2   		                            &        2010-04-20         &    4.8/4.8/4.8	& 1.0/0.9/0.9		& 1.2--1.3  	& 0.6   	& 34      	&  19.2     & 0.624 		& (9) \\
GRB100418A-3   		                            &        2010-04-21         &    4.8/4.8/4.8	& 1.0/0.9/0.9		& 1.2--1.4  	& 0.7   	& 58      	&   >24    	& 0.624 		& (9) \\
GRB100424A\tablefootmark{c} 	                &        2013-03-11         &    4.8/4.8/4.8	& 1.0/0.9/0.9		& 1.1--1.2  	& 0.9   	& 25239.1   &   >24    	& 2.465  		& (2) \\
GRB100425A     		                            &        2010-04-25         &    2.4/2.4/2.4	& 1.0/0.9/0.9		& 1.5--1.3  	& 0.7   	& 4      	&  20.6    	& 1.755  		& (2,3) \\
GRB100615A\tablefootmark{c}		                &        2013-03-05         &    4.8/4.8/4.8	& 1.0/0.9/0.9		& 1.0--1.1  	& 0.9   	& 23858.8   &   >24   	& 1.398  		& (2) \\
GRB100621A     		                            &        2010-06-21         &    2.4/2.4/2.4	& 1.0/0.9/0.9		& 1.3--1.4  	& 1.0   	& 7.1      	&      	    & 0.542  		& (2) \\
GRB100625A\tablefootmark{c}\tablefootmark{f}    &        2010-07-07         &    4.8/4.8/4.8	& 1.0/0.9/0.9		& 1.1--1.0  	& 0.8   	& 278.7    	&   >24	    & 0.452  		& (2) \\
GRB100724A\tablefootmark{a}\tablefootmark{d} 	&        2010-07-24         &    4.2/4.2/4.2	& 1.0/0.9/0.9		& 1.5--2.3  	& 0.7   	& 0.2      	&      	    & 1.288  		& (2) \\
GRB100728B\tablefootmark{e} 	                &        2010-07-29         &    7.2/7.2/7.2	& 1.0/0.9/0.9		& 1.5--1.1  	& 0.6   	& 22      	&  23    	& 2.106  		& (2) \\
GRB100814A-1\tablefootmark{d} 	                &        2010-08-14         &    0.9/0.9/0.9	& 1.0/0.9/0.9		& 1.9--1.7  	& 0.5   	& 0.9      	&  19    	& 1.439   		& (2) \\
GRB100814A-2   		                            &        2010-08-14         &    4.8/4.8/4.8	& 1.0/0.9/0.9		& 1.5--1.2  	& 0.7   	& 2.1      	&  19    	& 1.439   		& (2) \\
GRB100814A-3   		                            &        2010-08-18         &    4.8/4.8/4.8	& 1.0/0.9/0.9		& 1.2--1.0  	& 0.6   	& 98      	&  20    	& 1.439   		& (2) \\
GRB100816A\tablefootmark{f}		                &        2010-08-17         &    4.8/4.8/4.8	& 1.0/0.9/0.9		& 1.8--1.6  	& 0.8   	& 28.4      &        	& 0.805  		& (2) \\
GRB100901A     		                            &        2010-09-04         &    2.4/2.4/2.4	& 1.0/0.9/0.9		& 1.5--1.5  	& 1.9   	& 66      	&   >24    	& 1.408  		& (10) \\
GRB101219A     		                            &        2010-12-19         &    7.2/7.2/7.2	& 1.0/0.9/0.9		& 1.1--1.7  	& 1.8   	& 3.7      	&      	    & 0.718  		& (2) \\
GRB101219B-1\tablefootmark{a}                   &        2010-12-20         &    4.8/4.8/4.8	& 1.0/0.9/0.9		& 1.6--2.6  	& 1.4   	& 11.6      &  20    	& 0.552 		& (11) \\
GRB101219B-2\tablefootmark{a}                   &        2011-01-05         &    7.2/7.2/7.2	& 1.0/0.9/0.9		& 1.2--2.0  	& 1.0   	& 394      	&  22.7    	& 0.552 		& (11) \\
GRB101219B-3\tablefootmark{a}                   &        2011-01-25         &    7.2/7.2/7.2	& 1.0/0.9/0.9		& 1.4--2.1  	& 0.7   	& 886      	&   >24    	& 0.552 		& (11) \\
GRB110128A     		                            &        2011-01-28         &    7.2/7.2/7.2	& 1.0/0.9/0.9		& 2.0--1.6  	& 0.6   	& 5.5      	&  22.5    	& 2.339  		& (2) \\
GRB110407A     		                            &        2011-04-08         &    9.6/9.6/9.6	& 1.0/0.9/0.9		& 1.4--1.3  	& 2.1   	& 12.4      &  23    	&  --    		& (2) \\
GRB110709B\tablefootmark{c}  	                &        2013-03-19         &    7.2/7.2/7.2	& 1.0/0.9/0.9		& 1.6--1.1  	& 0.9   	& 14834.8   &   >24    	&  2.109 		& (2) \\
GRB110715A\tablefootmark{a}     		        &        2011-07-16         &    0.6/0.6/0.6	& 1.0/0.9/0.9		& 1.1--1.1  	& 1.6   	& 12.3      &  18.5    	& 0.823  		& (2) \\
GRB110721A\tablefootmark{a}     		        &        2011-07-22         &    2.4/2.4/2.4	& 1.0/0.9/0.9		& 1.2--1.4  	& 2.3   	& 28.7      &   >24    	& 0.382  		& (2) \\
GRB110808A     		                            &        2011-08-08         &   2.4/2.4/2.4 	& 1.0/0.9/0.9		& 1.2--1.1  	& 1.0   	& 3.0      	&  21.2    	& 1.3488 		& (2) \\
GRB110818A     		                            &        2011-08-19         &   4.8/4.8/4.8 	& 1.0/0.9/0.9		& 1.3--1.3  	& 0.9   	& 6.2      	&  22.3    	& 3.36   		& (2) \\
GRB111005A\tablefootmark{a}\tablefootmark{c}    &        2013-04-01         &   1.2/1.2/1.2 	& 1.0/0.9/0.9		& 1.3--1.3  	& 0.7   	& 13052.0   &   >24    	& 0.013? 		& (2) \\
GRB111008A-1   		                            &        2011-10-09         &   8.8/8.8/8.4 	& 1.0/0.9/0.9		& 1.1--1.0  	& 1.3   	& 8.5      	&  21    	& 4.9898 		& (12) \\
GRB111008A-2   		                            &        2011-10-10         &   8.0/8.0/7.2 	& 1.0/0.9/0.9		& 1.3--1.0  	& 0.9   	& 20.1      &  22    	& 4.9898 		& (12) \\
GRB111107A     		                            &        2011-11-07         &   4.8/4.8/4.8 	& 1.0/0.9/0.9		& 1.8--1.5  	& 0.8   	& 5.3      	&  21.5    	& 2.893  		& (2) \\
GRB111117A\tablefootmark{f}		                &        2011-11-19         &   4.8/4.8/4.8 	& 1.0/0.9/0.9		& 1.5--1.4  	& 0.7   	& 38      	&   >24    	& 1.3?   		& (2) \\
GRB111123A-1   		                            &        2011-11-24         &   6.2/6.6/6.6 	& 1.0/0.9/0.9		& 1.6--1.1  	& 0.8   	& 12.2      &   >24    	& 3.1516 		& (2) \\
GRB111123A-2\tablefootmark{c} 	                &        2013-03-07         &    2.4/2.4/2.4	& 1.0/0.9/0.9		& 1.0--1.0  	& 0.5   	& 11266.1   &   >24    	& 3.1516 		& (2) \\
GRB111129A     		                            &        2011-11-30         &   3.6/3.6/3.6 	& 1.0/0.9/0.9		& 1.6--2.1  	& 1.9   	& 8.7      	&  >24 	    & 1.080    		& (2) \\
GRB111209A-1   		                            &        2011-12-10         &   4.8/4.8/4.8 	& 1.0/0.9/0.9		& 1.1--1.2  	& 0.8   	& 17.7      &  20.1    	& 0.677  		& (13) \\
GRB111209A-2   		                            &        2011-12-29         &   9.6/9.6/9.6 	& 1.0/0.9/0.9		& 1.2--2.0  	& 1.0   	& 497      	&  23    	& 0.677  		& (13) \\
GRB111211A\tablefootmark{a}  		            &        2011-12-13         &   2.4/2.4/2.4 	& 1.0/0.9/0.9		& 1.4--1.6  	& 0.6   	& 31      	&  19.5    	& 0.478  		& (2) \\
GRB111228A     		                            &        2011-12-29         &   2.4/2.4/2.4 	& 1.0/0.9/0.9		& 1.4--1.4  	& 0.7   	& 15.9      &  20.1    	& 0.716  		& (2) \\
GRB120118B\tablefootmark{c} 		            &        2013-02-13         &   3.6/3.6/3.6 	& 1.0/0.9/0.9		& 1.1--1.0  	& 0.7   	& 9393.4    &   >24    	& 2.943  		& (2) \\
GRB120119A-1   		                            &        2012-01-19         &   2.4/2.4/2.4 	& 1.0/0.9/0.9		& 1.1--1.1  	& 0.6   	& 1.4      	&  17    	& 1.728  		& (2) \\
GRB120119A-2   		                            &        2012-01-19         &   1.2/1.2/1.2 	& 1.0/0.9/0.9		& 1.8--1.9  	& 0.5   	& 4.5      	&  20    	& 1.728  		& (2) \\
GRB120119A-3\tablefootmark{c} 	                &        2013-02-26         &    4.8/4.8/4.8	& 1.0/0.9/0.6JH 	& 1.0--1.1  	& 1.8  	    & 9693.9    &   >24    	& 1.728  		& (2) \\
GRB120211A-1\tablefootmark{c}   		        &        2013-02-17         &   4.8/4.8/4.8   	& 1.0/0.9/0.9       & 1.1--1.4      & 1.3   	& 8918.7    &   >24     & 2.346 		& (2) \\
GRB120211A-2\tablefootmark{c}   		        &        2013-03-20         &   3.6/3.6/3.6   	& 1.0/0.9/0.9       & 1.1--1.2      & 1.2   	& 9660.3    &   >24     & 2.346 		& (2) \\
GRB120224A     		                            &        2012-02-25         &   2.4/2.4/2.4 	& 1.0/0.9/0.9		& 1.7--2.1  	& 1.3   	& 19.8      &  22.3    	& 1.10    		& (2) \\
GRB120311A\tablefootmark{a}      	            &        2012-03-11         &   2.4/2.4/2.4 	& 1.0/0.9/0.9		& 1.6--1.4  	& 0.7   	& 3.7      	&  21.6    	& 0.350    		& (2) \\
GRB120327A-1\tablefootmark{a}   		        &        2012-03-27         &   2.4/2.4/2.4 	& 1.0/0.9/0.9		& 1.6--1.4  	& 0.6   	& 2.1      	&  18.8    	& 2.815  		& (14) \\
GRB120327A-2\tablefootmark{a}   		        &        2012-03-28         &   4.2/4.2/4.2 	& 1.0/0.9/0.9		& 1.0--1.1  	& 0.6   	& 29      	&  22.5    	& 2.815  		& (14) \\
GRB120404A     		                            &        2012-04-05         &   9.6/9.6/9.6 	& 1.0/0.9/0.9JH 	& 1.7--1.3 		& 1.3  	    & 15.7      &  21.3    	& 2.876  		& (2) \\
GRB120422A   		                            &        2012-04-22         &   4.8/4.8/4.8 	& 1.0/0.9/0.9		& 1.3--1.3  	& 0.7   	& 16.5      &  22    	& 0.283  		& (15) \\
GRB120712A     		                            &        2012-07-13         &   4.8/4.8/4.8 	& 1.0/0.9/0.9 	    & 1.5--2.5 		& 1.5   	& 10.4      &  21.5    	& 4.175  		& (2) \\
GRB120714B     		                            &        2012-07-15         &   4.8/4.8/4.8 	& 1.0/0.9/0.9JH 	& 1.5--1.2 		& 1.2   	& 7.8     	&  22.1    	& 0.398  		& (2)\\
GRB120716A\tablefootmark{a}  		            &        2012-07-19         &   3.6/3.6/3.6 	& 1.0/0.9/0.9JH 	& 1.8--2.6 		& 1.1   	& 62      	&  20.9    	& 2.486  		& (2) \\
GRB120722A\tablefootmark{b} 		            &        2012-07-22         &   4.8/4.8/4.8 	& 1.0/0.9/0.9 	    & 1.3--1.3 		& 1.2   	& 10.3      &  23.6    	& 0.959  		& (2) \\
GRB120805A\tablefootmark{b} 		            &        2012-08-14         &   3.6/3.6/3.6 	& 1.0/0.9/0.9JH 	& 1.3--1.7 		& 0.9   	& 218      	&   >24    	& 2.8?   		& (2) \\
GRB120815A\tablefootmark{a}     		        &        2012-08-15         &   2.4/2.4/2.4 	& 1.0/0.9/0.9 	    & 1.3--1.4 		& 0.7   	& 1.69      &  18.9     & 2.358  		& (16) \\
GRB120909A\tablefootmark{d}                     &        2012-09-09         &   1.2/1.2/1.2 	& 1.0/0.9/0.9 	    & 1.6--1.6 		& 1.6   	& 1.7     	&  21    	& 3.929  		& (2) \\
GRB120923A     		                            &        2012-09-23         &   9.6/9.6/9.6 	& 1.0/0.9/0.9JH 	& 1.2--1.4 		& 1.0   	& 18.5      &   >24    	& $\gtrsim8$ 	& (2) \\
GRB121024A     		                            &        2012-10-24         &   2.4/2.4/2.4 	& 1.0/0.9/0.9 	    & 1.2--1.1 		& 0.6   	& 1.8     	&  20    	& 2.300  		& (17) \\
GRB121027A     		                            &        2012-10-30         &   8.4/8.4/8.4 	& 1.0/0.9/0.9 	    & 1.3--1.3 		& 1.3   	& 69.4      &  21.15    & 1.773  		& (2) \\
GRB121201A     		                            &        2012-12-02         &   4.8/4.8/4.8 	& 1.0/0.9/0.9JH 	& 1.1--1.1 		& 1.1   	& 12.9      &  23    	& 3.385  		& (2) \\
GRB121229A     		                            &        2012-12-29         &   4.8/4.8/4.8 	& 1.0/0.9/0.9JH 	& 1.4--1.2 		& 1.5   	& 2     	&  21.5    	& 2.707  		& (2) \\
GRB130131B\tablefootmark{c} 		            &        2013-03-09         &   7.2/7.2/7.2 	& 1.0/0.9/0.9JH 	& 1.3--1.6 		& 1.1   	& 874.1     &   >24    	& 2.539  		& (2) \\
GRB130408A\tablefootmark{a}     		        &        2013-04-08         &   1.2/1.2/1.2 	& 1.0/0.9/0.9 	    & 1.0--1.0 		& 0.9   	& 1.9     	&  20    	& 3.758  		& (2) \\
GRB130418A     		                            &        2013-04-18         &   1.2/1.2/1.2 	& 1.0/0.9/0.9 	    & 1.4--1.3 		& 1.2   	& 4.6     	&  18.5    	& 1.222  		& (2) \\
GRB130427A     		                            &        2013-04-28         &   1.2/1.2/1.2 	& 1.0/0.9/0.9JH 	& 1.8--1.8 		& 0.8   	& 16.5      &  19    	& 0.340  		& (18) \\
GRB130427B     		                            &        2013-04-28         &   1.2/1.2/1.2 	& 1.0/0.9/0.9JH 	& 1.2--1.0 		& 1.0   	& 20.3      &  22.7    	& 2.780   		&  (2) \\
GRB130603B\tablefootmark{f}		                &        2013-06-04         &   2.4/2.4/2.4 	& 1.0/0.9/0.9 	    & 1.4--1.4 		& 1.1   	& 8.2     	&  21.5    	& 0.356  		& (19) \\
GRB130606A     		                            &        2013-06-07         &   4.2/4.2/4.2 	& 1.0/0.9/0.9JH 	& 1.7--1.9 		& 0.9   	& 7.1     	&  19    	& 5.91   		& (20) \\
GRB130612A     		                            &        2013-06-12         &   1.2/1.2/1.2 	& 1.0/0.9/0.9 	    & 1.3--1.3 		& 1.5   	& 1.1     	&  21.5    	& 2.006  		& (2) \\
GRB130615A     		                            &        2013-06-15         &   1.2/1.2/1.2 	& 1.0/0.9/0.9 	    & 2.1--2.2 		& 1.0   	& 0.8     	&  21    	& $\sim3$  		& (2) \\
GRB130701A     		                            &        2013-07-01         &   1.2/1.2/1.2 	& 1.0/0.9/0.9JH 	& 2.0--2.0 		& 1.4   	& 5.5     	&  19.9    	& 1.155  		& (2) \\
GRB130925A                                      &        2013-09-25         &   5.88/6.0/6.9     & 1.0/0.9/0.9JH     & 1.0--1.0      & 0.6       & 3.5       &       & 0.347         & (2) \\

GRB131011A\tablefootmark{a}			            &        2013-10-13         &   4.5/4.5/4.5 	& 1.0/0.9/0.9		& 1.1--1.1		& 0.8 	    & 34.2     	&   >24   	& 1.874			& (2) \\
GRB131030A			                            &        2013-10-31         &   3.6/3.6/3.6 	& 1.0/0.9/0.9		& 1.1--1.1		& 1.1 	    & 3.4     	&  18.0	    & 1.296			& (2) \\
GRB131103A			                            &        2013-11-05         &   2.4/2.4/2.4 	& 1.0/0.9/0.9JH		& 1.1--1.1		& 1.0 	    & 5.8     	&  20.48   	& 0.599			& (2) \\
GRB131105A			                            &        2013-11-05         &   4.8/4.8/4.8 	& 1.0/0.9/0.9		& 1.3--1.4		& 0.8 	    & 1.3     	&    22.4  	& 1.686			& (2) \\
GRB131117A			                            &        2013-11-17         &   4.8/4.8/4.8 	& 1.0/0.9/0.9JH		& 1.3--1.2		& 1.7 	    & 1.1     	&  20	    & 4.042			& (2) \\
GRB131231A\tablefootmark{a}			            &        2014-01-01         &   2.4/2.4/2.4 	& 1.0/0.9/0.9JH		& 1.4--1.3		& 0.9 	    & 20.2     	&    18.5  	& 0.642			& (2) \\
GRB140114A\tablefootmark{c}                     &        2014-03-28         &   5.4/5.4/5.4 	& 1.0/0.9/0.9JH		& 1.7--1.7		& 1.2 	    & 1745.7    &  >24  	& ~2.8			& (2) \\
GRB140213A\tablefootmark{a}                     &        2014-02-14         &   1.2/1.2/1.2 	& 1.0/0.9/0.9JH		& 1.5--1.5		& 0.7 	    & 5.8     	&   19.5  	& 1.208			& (2) \\
GRB140301A			                            &        2014-03-02         &   7.2/7.2/7.2 	& 1.0/0.9/0.9JH		& 1.1--1.1		& 0.9 	    & 9     	&    23.1  	& 1.416			& (2) \\
GRB140311A\tablefootmark{a}                     &        2014-03-13         &   7.6/6.3/8.4 	& 1.0/0.9/0.9JH		& 1.2--1.2		& 0.6 	    & 32.5     	&  >24  	& 4.954			& (2) \\
GRB140430A\tablefootmark{a}                     &        2014-04-30         &   1.2/1.2/1.2 	& 1.0/0.9/0.9		& 2.0--1.8		& 1.6 	    & 2.5     	&  19	    & 1.601			& (2) \\
GRB140506A-1		                            &        2014-05-07         &   4.8/4.8/4.8 	& 1.0/0.9/0.9		& 1.3--1.4		& 0.7 	    & 8.8     	&   20.9  	& 0.889			& (2) \\
GRB140506A-2		                            &        2014-05-08         &   4.8/4.8/4.8 	& 1.0/0.9/0.9		& 1.2--1.3		& 0.7 	    & 32.9     	&   >24   	& 0.889			& (2) \\
GRB140515A			                            &        2014-05-16         &   4.8/4.8/4.8 	& 1.0/0.9/0.9		& 1.3--1.3		& 1.4 	    & 15.5     	&     	    & ~6.32			& (2) \\
GRB140614A			                            &        2014-06-14         &   2.4/2.4/2.4 	& 1.0/0.9/0.9		& 1.8--1.8		& 0.7 	    & 3.8     	&   21.5   	& 4.233			& (2) \\
GRB140622A			                            &        2014-06-22         &   1.2/1.2/1.2 	& 1.0/0.9/0.9		& 1.4--1.3		& 1.0 	    & 0.8     	&           & 0.959			& (2) \\
GRB141028A\tablefootmark{a}                     &        2014-10-29         &   2.4/2.4/2.4 	& 1.0/0.9/0.9		& 1.5--1.4		& 1.0 	    & 15.4     	&   20 	    & 2.332			& (2) \\
GRB141031A\tablefootmark{a}\tablefootmark{c}    &        2015-01-29         &   2.4/2.4/2.4 	& 1.0/0.9/0.9		& 1.2--1.3		& 0.8 	    & 10911.8   &   >24   	& 				& (2) \\
GRB141109A-1		                            &        2014-11-09         &   2.4/2.4/2.4     & 1.0/0.9/0.9JH     & 1.5--1.7      & 0.8       & 1.9       &   19.2    & 2.993         & (2) \\
GRB141109A-2		                            &        2014-11-10         &   4.3/4.3/4.5     & 1.0/0.9/0.9JH     & 1.7--2.0      & 0.8       & 25.4      &           & 2.993         & (2) \\
GRB150206A\tablefootmark{a}                     &        2015-02-07         &   2.4/2.4/2.4     & 1.0/0.9/0.9       & 2.1--1.9      & 0.8       & 10        &  21.9     & 2.087         & (2) \\
GRB150301B			                            &        2015-03-02         &   3.6/3.6/3.6     & 1.0/0.9/0.9JH     & 1.2--1.2      & 1.1       & 5.1       &           & 1.5169        & (2) \\
GRB150403A			                            &        2015-04-04         &   2.4/2.4/2.4     & 1.0/0.9/0.9       & 1.6--1.7      & nan       & 10.8      &  19.1     & 2.06          & (2) \\
GRB150423A\tablefootmark{d}\tablefootmark{f}    &        2015-04-23         &   4.8/4.8/4.8     & 1.0/0.9/0.9       & 2.7--2.4      & 1.4       & 0.4       &           & 1.394         & (2) \\
GRB150428A			                            &        2015-04-28         &   2.4/2.4/2.4     & 1.0/0.9/0.9JH     & 1.6--1.5      & 0.8       & 3.7       &           &               & (2) \\
GRB150514A\tablefootmark{a}                     &        2015-05-15         &   2.4/2.4/2.4     & 1.0/0.9/0.9       & 2.3--2.1      & 0.9       & 28.4      &  19.5     & 0.807         & (2) \\
GRB150518A\tablefootmark{a}                     &        2015-05-20         &   2.4/2.4/2.4     & 1.0/0.9/0.9JH     & 1.3--1.3      & 1.7       & 30.7      &  >24      & 0.256         & (2) \\
GRB150616A-1\tablefootmark{a}                   &        2015-06-17         &   2.4/2.4/2.4     & 1.0/0.9/0.9       & 1.4--1.5      & 0.8       & 5.7       &           & 0.988          & (2) \\
GRB150616A-2\tablefootmark{a}\tablefootmark{c}  &        2015-09-12         &   2.4/2.4/2.4     & 1.0/0.9/0.9JH     & 1.2--1.1      & 1.2       & 2091.9    &           & 0.988    & (2) \\
GRB150727A			                            &        2015-07-28         &   3.6/3.6/2.4 	& 1.0/0.9/0.9JH		& 1.2--1.2		& 1.4 	    & 5.0     	&           & 0.313 		& (2) \\
GRB150821A\tablefootmark{d}                     &        2015-08-21         &   2.4/2.4/2.4 	& 1.0/0.9/0.9		& 2.0--1.8		& 1.3 	    & 0.2     	& 16     	& 0.755  		& (2) \\
GRB150910A			                            &        2015-09-11         &   1.8/1.8/1.8 	& 1.0/0.9/0.9JH		& 1.9--1.9		& 1.3 	    & 20.1     	&           & 1.359    		& (2) \\
GRB150915A			                            &        2015-09-16         &   4.8/4.8/4.8 	& 1.0/0.9/0.9JH		& 1.1--1.1		& 1.6 	    & 3.3     	& 23   	    & 1.968   		& (2) \\
GRB151021A\tablefootmark{d}                     &        2015-10-21         &   4.2/4.2/4.2 	& 1.0/0.9/0.9		& 1.0--1.1		& 1.4 	    & 0.75     	& 18.2     	& 2.33    		& (2) \\
GRB151027B			                            &        2015-10-28         &   2.4/2.4/2.4 	& 1.0/0.9/0.9JH		& 1.5--1.7		& 1.2 	    & 5     	& 20.5     	& 4.063   		& (2) \\
GRB151029A			                            &        2015-10-29         &   1.2/1.2/1.2 	& 1.0/0.9/0.9JH		& 1.9--1.7		& 1.1 	    & 1     	& 20   	    & 1.423 		& (2) \\
GRB151031A			                            &        2015-10-31         &   4.2/4.2/4.2 	& 1.0/0.9/0.9		& 1.1--1.1		& 1.1 	    & 0.3     	& 20.4     	& 1.167   		& (2) \\
GRB160117B			                            &        2016-01-18         &   4.8/4.8/4.8 	& 1.0/0.9/0.9JH		& 1.1--1.2		& 1.1 	    & 13.5     	& 21   	    & 0.87   		& (2) \\
GRB160203A\tablefootmark{d}                     &        2016-02-03         &   6.6/6.6/6.6 	& 1.0/0.9/0.9		& 1.0--1.8		& 1.0 	    & 0.3     	& 18   	    & 3.52   		& (2) \\
GRB160228A\tablefootmark{c}                     &        2016-03-12         &   4.8/4.8/4.8 	& 1.0/0.9/0.9JH		& 1.7--1.7		& 1.0 	    & 295.8     & >24    	& 1.64  		& (2) \\
GRB160303A\tablefootmark{f}                     &        2016-03-04         &   4.8/4.8/4.8 	& 1.0/0.9/0.9JH		& 1.6--1.5		& 0.8 	    & 19.1     	&           &         		& (2) \\
GRB160314A			                            &        2016-03-15         &   4.8/4.8/4.8 	& 1.0/0.9/0.9JH		& 1.3--1.3		& 0.8 	    & 13.0     	&           & 0.726    		& (2) \\
GRB160410A\tablefootmark{d}\tablefootmark{f}    &        2016-04-10         &   1.8/1.8/1.8 	& 1.0/0.9/0.9		& 2.5--2.3		& 0.5 	    & 0.15     	& 20.3    	& 1.717   		& (2) \\
GRB160425A			                            &        2016-04-26         &   4.8/4.8/4.8 	& 1.0/0.9/0.9JH		& 1.3--1.3		& 0.5 	    & 7.2     	&           & 0.555 		& (2) \\
GRB160625B\tablefootmark{a}                     &        2016-06-27         &   2.4/2.4/2.4 	& 1.0/0.9/0.9JH		& 1.3--1.3		& 0.7 	    & 30     	& 19.1   	& 1.406 		& (2) \\
GRB160804A-1\tablefootmark{a}   		        &        2016-08-04         &   2.4/2.4/2.4  	& 1.0/0.9/0.9JH 	& 1.4--1.3 		& 0.6  	    & 22.4      & 21.2   	& 0.736  		& (2) \\
GRB160804A-2\tablefootmark{a}\tablefootmark{c}  &        2016-08-27         &   3.6/3.6/3.6  	& 1.0/0.9/0.9JH 	& 1.9--1.8 		& 0.6  	    & 574.4     &   >24   	& 0.736   		& (2) \\
GRB161001A			                            &        2016-10-01         &   2.4/2.4/2.4 	& 1.0/0.9/0.9JH		& 1.2--1.3		& 0.5 	    & 6.1     	&           & 0.891  		& (2) \\
GRB161007A\tablefootmark{c}  	                &        2016-10-14         &   2.4/2.4/2.4  	& 1.0/0.9/0.9JH 	& 1.6--1.6 		& 0.7   	& 323      	&   >24   	&       		& (2) \\
GRB161014A   		                            &        2016-10-15         &   4.8/4.8/4.8  	& 1.0/0.9/0.9JH 	& 1.1--1.2 		& 0.5   	& 11.6      &  21.4   	& 2.823   		& (2) \\
GRB161023A\tablefootmark{a}			            &        2016-10-24         &   1.2/1.2/1.2 	& 1.0/0.9/0.9JH		& 1.2--1.2		& 0.9 	    & 3     	&  17.5   	& 2.710 		& (2) \\
GRB161117A                                      &        2016-11-17         &   2.4/2.4/2.4     & 1.0/0.9/0.9       & 1.8--1.6      & 2.6       & 0.73      &  19       & 1.549         & (2) \\
GRB161219B                                      &        2016-12-21         &   2.4/2.4/2.4     & 1.0/0.9/0.9JH     & 1.1--1.1      & 0.9       & 35.7      &           & 0.1475        & (2) \\










































































































































\hline\noalign{\smallskip}


\end{longtable}
\centering
\begin{minipage}{5.3in}
\tablefoot{
\tablefoottext{a}{Not part of the statistical sample}
\tablefoottext{b}{Spectrum dominated by light from the host galaxy}
\tablefoottext{c}{Spectrum of the host galaxy taken long after the burst}
\tablefoottext{d}{RRM observation}
\tablefoottext{e}{ADC malfunction during observation}
\tablefoottext{f}{Short burst}
}

%\tablebib{}
\end{minipage}
\end{longtab}
	%-------------------------END TABLE---------------------------------------
